\documentclass{article}
\usepackage{amsmath, amsfonts, amssymb}

\author{Matteo Paz, Dylan Rupel}
\date{January 8, 2024}
\title{Combinatorics}

\begin{document}
\maketitle{}
\section{Counting Organized Data - The Twelvefold Way}
\emph{\underline{Goal}} - Count the number of functions between multi-sets (Some elements may be identical).
\begin{itemize}
    \item How many surjective functions exist between $\{1,2,3,4,5\}$ to $\{a,b,c\}$?
\end{itemize}
We could try to list them all, but there should be a way to take advantage of the structure. Given that 
we are considering a multi-set, lets generalize to: \\
$\{1,2,3,4,5\} \to \{\cdot, \cdot, \cdot\}$ Where each dot can be thought of as a bucket. It doesnt matter necessarily which bucket
a number is being placed in, but how many and which went in each of the 3 baskets.
\vspace{1em}

\subsection{The Twelvefold Way}

There turns out to be 12 major classifications of mappings between sets. The number of such functions for each classification is a
very useful tool when counting. Let us map
$N \to X$, with $|N| = n, |X| = x$.

$\begin{array}{c|c|c|c}
& \text{Functions} & \text{Injective} & \text{Surjective} \\ \hline
\substack{\text{Distinguishable Domain and Codomain}} & x^n & x^{\underline{n}} & x!{n \brace x} \\ \hline
\substack{\text{Indistinguishable Domain, Dist. Codomain}} & {n + x - 1 \choose x - 1} & {x \choose n} & \\ \hline
\substack{\text{Dist. Domain, Indist. Codomain}} & \sum^x_{k=1} {n \brace k} & [n \leq x] & {n \brace x} \\ \hline
\substack{\text{Indist. Domain and Codomain}} & & [n \leq x] & \\ \hline
\end{array}$

\subsection*{Distinguishable Functions}
The top-left entry is $x^n$. This is because in identifying a function of this type, you have $x$ choices of a preimage for $n$ choices of an image.
\subsection*{Distinguishable and Injective Functions}
Very similar to non-injective functions, but every output must be unique. Thus having $x$ choices for your first element to map, then
$x-1$, $n$ times gives $x(x-1)\cdots(x-n+1)$ choices in all. This is the falling factorial function $x^{\underline{n}}$
\subsection*{Injective and Indistinguishable Functions}
The notion of an indistinguishable domain or codomain is that if you mess around with order or specificity, it makes no difference. So 
we should look at the ways you can permute either of the sets. Let $S_N$ denote the group of permutations on $N$. Let $X^N$ denote the set of all functions $F: N \to X$.
\vspace{0.5em}

Define a group action $S_N \circlearrowright X^N$ by $(\sigma \cdot f)(n) = f(\sigma^{-1}(n))$. Notice that this is in fact an action as associativity holds:
\[((\sigma \circ \tau) \cdot f)(n) = f(\tau^{-1}(\sigma^{-1}(n))) = (\sigma \cdot (\tau \cdot f))(n)\]
And $\text{Id} \in S_N$ works as an identity in this setting as well.
\vspace{0.5em}
Lets prove that $S_N \circlearrowright \text{Inj}(X^N)$ is a free action. If $f \in X^N$ is injective
and $\sigma \cdot f = f$, then $(\sigma \cdot f)(n) = f(n)$. 
\[f(\sigma^{-1}(n)) = f(n)\]
\[n = \sigma(n)\]
\[\sigma = \text{Id}\]
Thus $\text{Stab}_{S_N}(f) = \{Id\}$ and the action is free. By the Orbit-Stabilizer theorem,
\[\left| {\text{Inj}(X^N)} / {S_N} \right| = \frac{x^{\underline{n}}}{n!}\]
This can be rewritten as ${x \choose n}$.
\vspace*{1em}

Finally, for injective functions with an indistinguishable codomain, it really doesnt matter where 
you map $N$, as long as you can without overlap. Thus, for the last two rows, we have the condition that the
codomain has enough room for $X: [n \leq x]$

\section{Surjective functions}
Start with a distinguishable domain $N$ and indistinguishable $X$. The order in which you group $N$ to go to $X$ does not matter, just that
there are $x$ nonempty groups that then map to different baskets of $X$. Since the entirety of $N$ must be included as the domain, we know that we
must partition $N$ into $x$ partitions, which is given by this notation ${n \brace x}$. These are called stirling numbers of the second kind.
\vspace*{0.5em}

If we have a distinguishable codomain once again, we can just multiply by our options of order: $x!$. Thus the
number of surjective functions between sets is $x! {n \brace x}$. \vspace*{0.5em}

Carrying over to all functions from dist $N$ to indist $X$, not just surjective ones, is equivalent to partitioning the set
in a number of different ways, from one element in $X$ to all elements in $X$. Therefore we sum the ways to partition the domain:
\[\sum^n_{k=1} {n \brace k}\]


\section{Indistinguishable Domain and Codomain}

This row has the least amount of structure. The injective functions were simple to think about,
but regular functions and surjective functions have very little structure, and thus are hard
to give a closed form for. In a similar way to the previous row, if we can find the number
of surjective functions onto $x$ elements, we can simply sum this value up for all possible sizes
of a function's image from 1 to $x$ to count how many functions there are in total.
\vspace{1em}
\subsection{Partitions}
A \emph{partition} of $n \in \mathbb{N}$ is a decreasing sequence $a_1, \cdots, a_k \in \mathbb{N}$
such that $\sum^k_{i=0} a_i = n$.

\vspace{1em}
If we consider surjective mappings from $n$ to $x$ indistinguishable elements, the only way to
identify a unique function is by the sizes of the subsets in its preimage partition. For $k=x$ elements, we identify
that these functions have a bijection with the partitions of $n$ into $x$ elements, because all $n$ must
be grouped into one of the preimages and there are $x$ elements that the preimages are generated from.

\end{document}
