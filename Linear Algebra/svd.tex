\documentclass{article}
\usepackage{amsmath, amsfonts}
\usepackage{base}


\author{Matteo Paz, Dylan Rupel}
\date{Apr 26th, 2024}
\title{The Singular Value Decomposition}

\begin{document}
    \maketitle{}
    \noindent

    First, we need a matrix. Let $A \in \operatorname{Mat}_n(\CC)$ be Hermetian. Therefore, we can find an \emph{orthonormal basis of eigenvectors}. There is a small issue with this, what if we want this for a non-square matrix? \\
    \textbf{Goal: Be able to do the same thing if $A$ is not square.} \vspace{1em}

    Assume that $\bfw_1, \ldots, \bfw_n$ is an orthonormal basis for $\CC^n$, and $A$ not necessarily hermitian. What properties must there be of $\bfw_i$ if $A\bfw_1, \ldots, A\bfw_n$ is still an orthogonal set?
    \[\langle A\bfw_i, A\bfw_j \rangle = 0\]
    \[\langle A^H A\bfw_i, \bfw_j \rangle = 0\]
    This would work if $\bfw_i$ was an eigenvector for $A^H A$. Let $B = A^H A$. Notice that $B$ is hermetian.
    \[B^H = (A^H A)^H = \overline{(\overline{A}^T A)}^T = (A^T \overline{A})^T = A^H A = B\]
    Then we would just want an orthonormal set of eigenvectors of $B$, which since $B$ is hermetian, does exist. Thus, set those original $\bfw_1, \ldots, \bfw_n$ to the orthonormal set of eigenvectors for $B$, with eigenvalues $\lambda_1, \ldots, \lambda_n$ respectively. This would imply that $A\bfw_1, \ldots, A\bfw_n$ is again orthogonal. But would it be orthonormal?
    \[|A\bfw_i| = \sqrt{\langle A\bfw_i, A\bfw_i \rangle}\]
    \[|A\bfw_i| = \sqrt{\langle A^H A\bfw_i, \bfw_i \rangle}\]
    \[|A\bfw_i| = \sqrt{\langle B\bfw_i, \bfw_i \rangle}\]
    \[= \sqrt{\lambda_i \langle \bfw_i, \bfw_i \rangle}\]
    \[= \sqrt{\lambda_i}\]
    Thus the magnitude of each basis vector after application of $A$ is the square root of its eigenvalue under $A^H A$. These magnitudes are the \textbf{singular values}. We denote $\sigma_i = \sqrt{\lambda_i}$. \vspace{1em}

    Define $\bfv_i = \frac{A\bfw_i}{\sigma_i}$ and let $\bfv_{n+1}, \ldots, \bfv_{m}$ be an orthonormal basis for $\Nul(A^T)$, which is perpendicular to the image of $A$. Set
    \[U = \begin{bmatrix}
        \bfw_1 | & \cdots & | \bfw_n
    \end{bmatrix}\]
    \[V = \begin{bmatrix}
        \bfv_1 | & \cdots & | \bfv_m
    \end{bmatrix}
    \]
    Now, notice that $AU = \begin{bmatrix}
        A\bfw_1 \cdots A\bfw_n
    \end{bmatrix} = \begin{bmatrix}
        \sigma_1\bfv_1 \cdots \sigma_n\bfv_n
    \end{bmatrix} = V \begin{bmatrix}
        \operatorname{diag}(\sigma_1, \ldots, \sigma_n) \\
        \mathbf{0}
    \end{bmatrix}$
    Let $S =\begin{bmatrix}
        \operatorname{diag}(\sigma_1, \ldots, \sigma_n) \\
        \mathbf{0}
    \end{bmatrix}$. Then we have $AU = VS$.
    \begin{align}
        A = VSU^H
    \end{align}
    Now, we have a factorization where:
    \begin{enumerate}
        \item $V$ is an orthonormal basis for $\CC^m$, where the first $n$ columns are the orthonormal results of multiplication by $B= A^H A$.
        \item $S$ is a modified diagonal matrix of the singular values of $A$
        \item $U$ is the orthonormal eigenbasis generated by $B = A^H A$
    \end{enumerate}
    This is the singular value decomposition.








\end{document}