\documentclass{article}
\usepackage{amsmath, amsfonts}

\newtheorem{Def}{Definition}
\newtheorem{Proof}{Proposition}

\author{Matteo Paz, Dylan Rupel}
\date{December 7, 2023}
\title{Exponential Random Variables}

\begin{document}
\maketitle

\section{Formulation}
\begin{Def}
A nonnegative random variable $X$ is \underline{memoryless} if
\[P\{X > s+t | X > t\} = P \{X > s\}\]
\end{Def}

Lets work with this. We know how to rewrite this conditional probability:
\[\frac{P\{X > s + t, X > t\}} {P\{X > t\}} = P\{X > s\}\]
Because $X> s+t$ implies that $X>t$ already,
\[P\{X > s+t\} = P\{X>s\}P\{X>t\}\]
Define $\overline{F}(x) = P\{X > x\}$. Then a memoryless random variable gives:
\[\overline{F}(s+t) = \overline{F}(s) \overline{F}(t)\]
\begin{Proof}
Let $g(x)$ be a continuous nonnegative function which satisfies $g(s+t) = g(s)g(t)$ for $s,t \in \mathbb{R}_{\geq 0}$.
Then $g(x)$ is an exponential function.
\end{Proof}
\textbf{Proof} \\
\[g(1) = g\left(n \cdot \frac{1}{n}\right) = g\left(\frac{1}{n}\right)^n \implies g\left(\frac{1}{n}\right) = g(1)^\frac{1}{n}\]
\[g(n) = g(1)^n\]
For all $n \in \mathbb{Q}$. However, any real number is the limit of a sequence of rational numbers.
Let $x \in \mathbb{R}$ and $(x_i)$ a sequence of rationals converging to $x$. Using the continuity of $g(x)$ to move the limit around...
\[g(x) = g\left(\lim_{i\to\infty} x_i\right) =\lim_{i\to\infty} g(x_i) = \lim_{i\to\infty} g(1)^{x_i} \]
\[= \left(g(1)\right)^{\lim_{i\to\infty} x_i} = g(1)^x\]

We can express this directly as 
\[g(x) = e^{-\lambda x}\]
where $\lambda = -\ln(g(1)) \square$
\vspace{4em}

Finally then, we can express $\overline{F}(x) = e^{-\lambda x}$ for some $\lambda \in \mathbb{R}$.
From this, we know the CDF is $F(x) = 1-e^{-\lambda x}$. The probability distribution function is then the derivative, giving:
\[f(x) = \frac{d}{dx} F(x) = \lambda e^{-\lambda x}\]



\end{document}
