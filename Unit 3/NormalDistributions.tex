\documentclass{article}
\usepackage{amsmath, amsfonts}
\author{Matteo Paz, Dylan Rupel}
\date{December 5, 2023}
\title{Normal Distributions}

\begin{document}
\maketitle{}

\section{Basic Formulation}
\subsection{An Experiment - Assumptions}
Picture a dartboard in the $xy$-plane with the bullseye at the origin. 
Let $(X,Y)$ be the location where the dart lands.
\vspace{0.25em}
Let's make some assumptions.
\begin{enumerate}
    \item \textbf{Rotational invariance}:
    The \emph{deviation} from the origin only depends of the distance from the origin and not on the direction.
    \item \textbf{Distribution}: $X$ and $Y$ are independent random variables.
\end{enumerate}
\vspace{0.25em}
\subsection{What kind of function is this?}
Let $\tilde f: \enspace \mathbb{R}^2 \to \mathbb{R}$ denote the probability density function.
Assumption (1) implies that $\tilde f (x,y) = g(x^2 + y^2)$ for some function $g$, with the squared sum representing
the invariance under rotation. Assumption (2) $\tilde f(x,y) \propto \tilde f(x,0)f(0,y)$ as independent random events
must multiply. If we reframe into the perspective of the single-variable function $g$, we have
\[g(x^2 + y^2) = g(x^2)g(y^2)\]
The kinds of functions that have this property are exponentials, so we can say:
\[g(t) = Ae^{-Bt} \qquad A, B \in \mathbb{R}_{> 0}\]
With the added factor of $-1$ because we are aiming for the center,
so the probability should be greatest at low distances from the center, rather than the opposite.
We can decide on the value of $A$ and $B$ by forcing the entire volume of the probability to $1$, as it should be:
\[1 = \iint_{\mathbb{R}^2} \tilde f(x,y) dx dy = \iint_{\mathbb{R}^2} g(x^2+y^2) dxdy\]

\subsection{Using the Jacobean}
To simplify this integral, we should respect the rotational symmetry and rewrite in terms of
polar coordinates, $r$ and $\theta$.
\[x = r \cos(\theta) \quad y = r \sin(\theta)\]
It's not possible to provide thorough justification for this particular step, but a useful tool in this
integration is the wedge operation $\wedge$. The key properties are:
\begin{itemize}
    \item We can replace $dxdy$ with $dx\wedge dy$
and vice versa without issue. 

    \item It is skew-symmetric: $dx \wedge dy$ = $- \left(dy \wedge dx\right)$
    
    \item $dx \wedge dx = 0$ (consequence of the previous property)
\end{itemize}

Lets use the wedge operator to replace $dxdy$ in the integral:
\begin{align*}
    dx\wedge dy &= (\cos(\theta)dr - r\sin(\theta)d\theta)\wedge(\sin(\theta)dr + r\cos(\theta)d\theta) \\
    &= r\cos^2(\theta)dr\wedge d\theta - r \sin^2(\theta) d\theta \wedge dr \\
    &= r\left(\cos^2(\theta) + \sin^2(\theta)\right) dr \wedge d\theta \\
    &= r dr \wedge d\theta
\end{align*}


Now we have our substitution, so rewriting our integral with new bounds $0 \leq \theta \leq 2\pi$ and $r \geq 0$ we have 

\[\int_{0}^{2\pi}\int_{0}^{\infty} g(r^2) \enspace r dr d\theta\]
Lets perform the calculation:
\begin{align*}
    &= 2\pi \int_{0}^\infty Ar e^{-Br^2} dr \\
    &=\left[\frac{-2\pi A}{2B} e^{-Br^2}\right]^\infty_0 \\
    &= \frac{\pi A}{B} = 1
\end{align*}
So $A = \frac{\pi}{B}$ and finally we have the probability distribution based on a single parameter $B$:
\[\tilde f(x,y) = \frac{B}{\pi} e^{-B(x^2 + y ^2)}\]
By assumption 2, we can factor this. Naturally we could have:
\[f(x) = \sqrt{\frac{B}{\pi}} e^{-Bx^2}\]
We don't necessarily yet know that this is a probability distribution. The proof of this is contained already in 
these notes, and the explicit procedure is left to the reader. \emph{Hint: Use the fact that the only nonzero idempotent is 1}.
However, once we prove that it is, this is the general form of the normal distribution.

\subsection{Expectation}

Lets compute $E[X]$ for random variable $X$ with probability density function $f(x) = \sqrt{\frac{B}{\pi}} e^{-Bx^2}$
\[\int_{\mathbb{R}} x f(x) dx = \int_{-\infty}^\infty \sqrt{\frac{B}{\pi}} x e^{-Bx^2}dx\]
\[\left[-\frac{1}{2\sqrt{B\pi}}e^{-Bx^2}\right]_{-\infty}^{\infty}\]
\[E[X] = 0\]


\end{document}
