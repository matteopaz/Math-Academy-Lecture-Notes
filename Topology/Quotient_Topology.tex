\documentclass{article}
\usepackage{amsmath, amsfonts}
\usepackage{base}

\author{Matteo Paz, Dylan Rupel}
\date{March 5, 2024}
\title{The Quotient Topology}

\begin{document}
\maketitle

\section{The Quotient Topology and Examples}
\begin{definition}
    Let $X$ be a topological space and suppose $\sim$ is an equivalence relation on $X$. Then there is a natural surjective function $f: X \to X / \sim$ onto the set of equivalence classes $[x]_\sim$. The \underline{quotient topology} is the finest topology for which $f$ is continuous. This can be achieved exactly by defining $U \subseteq X / \sim$ open if and only if $f^{-1}(U) = \{x \in X : [x] \in U\}$ is open in $X$.
\end{definition}

\begin{example}
    Consider the unit sphere $S_3$ in $\RR^3$, and the disjoint union of two unit discs $S'_2$ in $\RR^2$, after identifying the boundaries of the disc with $\sim$.
\end{example}

If we take the intersection of an open ball in $\RR^3$ with $S_3$, we get a 2 dimensional ball on the surface of $S_3$, which are exactly the open sets of $S'_2$ - 2D balls. By identifying the boundaries of the two discs, we connect the equator of the sphere. Thus, these spaces are homeomorphic, and $S_3$ is the quotient topology for $S'_2 \sqcup S'_2 / \sim$

\begin{example}
    Consider $[0,1] / \sim$ where $0 \sim 1$.
\end{example}

This essentially wraps the interval $[0,1]$ into a circle of circumference $1$. Open intervals on the image about $0$ become half open in the domain: $[0,a) \sqcup (b, 1]$.

\begin{example}
    Consider $\RR$. Take $\mathbb{Z}$ to be one equivalence class, identifying them all together, and leaving all non integers alone.
\end{example}

    Similar to example 2, this creates a circle for every interval $[z,z+1]$. However, each circle does not identify except at its point $z$. Thus it creates a wedge of infinitely many circles, all meeting only at $0$. Any open set about that $0$ point will include an open interval about every point in $\mathbb{Z}$ in the preimage.

\section{Quotient of a Topological Space Under a Group Action}

\begin{definition}
    An action of a group $G$ on a set $X$ is $*: G \times X \to X$ such that
    \begin{enumerate}
        \item $g.(h.x) = (gh). x$ for all $g,h \in G$ and $x \in X$ 
        \item $e . x = x$ for all $x \in X$
    \end{enumerate}
\end{definition}

\begin{definition}
    A group $G$ is a topological group if it has a topology such that with $g \ in G$, the maps $\phi_g(h) = gh$ and $\rho_g(h) = hg$ are continuous.
\end{definition}

Lets see some topological groups.
\begin{enumerate}
    \item To start, any discrete / finite group is a topological group. With the discrete topology, any function is continuous, because all preimages are open: $(\mathbb{Z}, +), (S_n, \circ), D_n, (\mathbb{Z}_n, +)$
    \item $GL_n(\RR), GL_n(\mathbb{C})$
    \item Any subgroup of a topological group
\end{enumerate}

\begin{definition}
    Let $G$ be a topological group and $X$ a topological space. Then $G$ has a continuous action on $X$ if there exists $*: G \times X \to X$ so that $\phi_g: X \to X$ is a continuous function.
\end{definition}

\end{document}