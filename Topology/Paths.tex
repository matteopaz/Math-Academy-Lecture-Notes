\documentclass{article}
\usepackage{amsmath, amsfonts}
\usepackage{base}

\author{Matteo Paz, Dylan Rupel}
\date{February 27th, 2024}
\title{Paths}


\begin{document}
\maketitle
\begin{definition}
    A path in a topological space $Y$ is a continuous mapping from $[0,1] \to Y$.
\end{definition}

Intuitively, a path is just a line or curve in your topological space. We define it to be homeomorphic
to the real interval $[0,1]$. If we have two paths $p_1$ and $p_2$, we can also concatenate or merge
them naturally, given that the endpoint $p_1(1) = p_2(0)$:
\[
p_1 * p_2 = 
\begin{cases} 
    p_1(2t) & \text{if} \quad 0 \leq t \leq \frac{1}{2} \\
    p_2(2t) & \text{if} \quad \frac{1}{2} \leq t \leq 1
\end{cases}
\]

Using this we can define a natural equivalence relation, in which all points which are path-connected
are considered as a group:
\begin{proposition}
    $x_1 \sim x_2$ if there exists a path $p: [0,1] \to X$ with $p(0) = x_1$ and $p(1) = x_2$ gives
    an equivalence relation
\end{proposition}
\pf
\begin{enumerate}
    \item Reflexive: For $x \in X$, the path $p(t) = x$ for all $t \in [0,1]$ satisfies.
    \item Symmetric: If $x \sim x'$ then there is a path $p: x \to x'$. If we define $p'(t) = 1-p$, then $p'(0) = x'$ and $p'(1) = x$
    \item Transitive: If $x \sim x'$ and $x' \sim x''$, then simply concatenate the respective paths with the aforementioned construction so that $p * p': x \to x''$
\end{enumerate}

Thus, path-connectedness is an equivalence relation. How does this relate to regular connectedness though?

\begin{proposition}
    If $X$ is path connected, then $X$ is connected.
\end{proposition}
\pf

Assume $X$ is disconnected and path connected, thus $X = A \sqcup B$ for disjoint open sets $A,B \subseteq X$. Let $x \in A$ and $y \in B$. Let there be a path
$p: x \to y$. Since paths are continuous functions, $p^{-1}(A) \sqcup p^{-1}(B) = [0,1]$ with the preimages being open. However, the interval $[0,1]$ is 
connected. Thus we reach a contradiction, and $X$ must then be connected. \done

\vspace{1em}

So, we can conclude that the condition of path-connectedness is a stronger statement than connetedness. However, we used a hidden lemma in that last proof:
\begin{proposition}
    $[0,1]$ is connected.
\end{proposition}
\pf Assume that $[0,1]$ is disconnected and is equal to the disjoint union of open sets $A,B$. WLOG assume that $1$ lies in $B$.
Let $x = \sup(A)$. Any interval around $x$ would contain points greater than $\sup(A)$. Therefore $x$ is not an open point for $A$, and $x \not \in A$. Also, any 
interval around $x$ would contain points less than $x$, which are thus in $A$. Therefore $x$ is not an open point for $B$ and $x \not in B$.
Thus $\sup(A)$ is in $X$ but not included in $A\sqcup B$. This is a contradiction, so $[0,1]$ is connected.
\vspace{1em}

\begin{definition}
    A topological space $X$ is locally path connected if every point has an open neighborhood which is path connected.
\end{definition}

\begin{theorem}
    Let $X$ be a locally path connected topological space. Then $X$ is path connected if and only if 
    $X$ is connected.
\end{theorem}
\pf
$\mathbf{\implies}$ \\
Already shown that path connectedness implies connected. \\

$\mathbf{\impliedby}$
Using the equivalence relation for paths earlier, we call such a space a
\emph{path component}, a set where all elements are reachable by path. Let $A$ be a path component.
Let $a \in A$. Then since $X$ is locally path connected, there is an open neighborhood $a \in U$ that is path connected.
Thus there exists a path from any point in $A$ to $a$, then from $a$ to any point in $U$. So all of $U$ is reachable from all of
$A$. Thus $U \subseteq A$. Thus $A$ is open.
\vspace{0.5em}

If there were more than one path component, e.g. any points were not reachable by path, then this would give an open partition of $X$, and $X$
could not be connected. Thus locally path connected and connected implies path connectedness. \done

\end{document}