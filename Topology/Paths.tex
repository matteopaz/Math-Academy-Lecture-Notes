\documentclass{article}
\usepackage{amsmath, amsfonts}
\usepackage{base}

\author{Matteo Paz, Dylan Rupel}
\date{February 27th, 2024}
\title{Paths}


\begin{document}
\maketitle
\begin{definition}
    A path in a topological space $Y$ is a continuous mapping from $[0,1] \to Y$.
\end{definition}

Intuitively, a path is just a line or curve in your topological space. We define it to be homeomorphic
to the real interval $[0,1]$. If we have two paths $p_1$ and $p_2$, we can also concatenate or merge
them naturally, given that the endpoint $p_1(1) = p_2(0)$:
\[
p_1 * p_2 = 
\begin{cases} 
    p_1(2t) & \text{if} \quad 0 \leq t \leq \frac{1}{2} \\
    p_2(2t) & \text{if} \quad \frac{1}{2} \leq t \leq 1
\end{cases}
\]

Using this we can define a natural equivalence relation, in which all points which are path-connected
are considered as a group:
\begin{proposition}
    $x_1 \sim x_2$ if there exists a path $p: [0,1] \to X$ with $p(0) = x_1$ and $p(1) = x_2$ gives
    an equivalence relation
\end{proposition}
\pf
\begin{enumerate}
    \item Reflexive: For $x \in X$, the path $p(t) = x$ for all $t \in [0,1]$ satisfies.
    \item Symmetric: If $x \sim x'$ then there is a path $p: x \to x'$. If we define $p'(t) = 1-p$, then $p'(0) = x'$ and $p'(1) = x$
    \item Transitive: If $x \sim x'$ and $x' \sim x''$, then simply concatenate the respective paths with the aforementioned construction so that $p * p': x \to x''$
\end{enumerate}

Thus, path-connectedness is an equivalence relation.

\end{document}